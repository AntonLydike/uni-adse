\documentclass{article}
\usepackage{microtype}
\usepackage[utf8]{inputenc} 
\usepackage[a4paper, total={6in, 9.6in}]{geometry}
\usepackage{enumitem}
\usepackage{amsmath}
\usepackage{amssymb}
\usepackage{fancyhdr}
\usepackage{xcolor}
\usepackage{tikz}
\usepackage{pgfplots}
\usepackage{svg}
\usepackage{graphicx}

\widowpenalties=4 10000 10000 150 0

% header / footer style
\pagestyle{fancy}
\fancyhf{}
\rhead{Ad-hoc und Sensornetze  SS20}
\lhead{Daniel Maquet, Leopold Luley, Anton Lydike}
\rfoot{Seite \thepage}

% define some basic colors
\definecolor{greeen}{RGB}{34,139,34}
\newcommand\red[1]{\textcolor{red}{#1}}
\newcommand\green[1]{\textcolor{greeen}{#1}}
\newcommand\blue[1]{\textcolor{blue}{#1}}


% define a task 
\newcommand\task[2]{\noindent\textbf{Aufgabe #1)\hfill \underline{\,\,\,\,\,\,}\,\,/#2p.}}
% and the total points for this sheet
\newcommand\pointsttl[1]{\noindent\textbf{Gesamtpunkte: \hfill \underline{\,\,\,\,\,\,}\,\,/#1p.}}


\newcommand\cfgtitle[1]{\title{\vspace{-1.5cm}Übungsblatt #1\\%
\begin{large} Übungsgruppe Boole \end{large}} \lfoot{Übungsblatt #1}\cfoot{Übungsgruppe Boole}}
\author{Daniel Maquet, Leopold Luley, Anton Lydike}

% configure title to be sheet 1
\cfgtitle{1}
\date{Sonntag 10.05.2020}

\begin{document}
    \maketitle
    \thispagestyle{fancy}

    \task{1}{8}
    \begin{enumerate}[label=\arabic*)]
        \item Organic Computing beschäftigt sich mit selbständigen Systemen, also Systemen die sich selbst konfigurieren und optimieren. Welche aber auch selbst Probleme erkennen und lösen. Dafür müssen sie ihre Umgebung wahrnehmen können und bestimmte Ereignisse zu einem bestimmten Grad vorraussehen können. 

        \item Ad-hoc und Sensornetzte sind beides Infrastrukturlose Netzwerke. Ad-hoc Netzwerke sind dabei spontan konstruierte Netzwerke, wobei die Verbindung von Endgeräten im Vordergrund steht (ID-centric networks). Sensornetze hingegen sind data-centric Netzwerke von selbständigen, vernetzten sensoren welche Daten aus ihrer Umgebung sammeln und verarbeiten oder weiterleiten.
        
        \item Ad-hoc und Sensornetze müssen meist Selbstorganisiert sein, da sie eine Dezentrale struktur haben. Auch ist es wichtig, optimale Routen durch solche netzwerke zu finden, um Batterielaufzeit zu erhöhen. 
        
        Da nicht immer qualifiziertes/trainiertes Personal vorhanden ist, sollten diese Netzwerke sich so weit wie möglich selbst konfigurieren und reparieren. All diese Aspekte machen sie aus der Perspektive von Organic Computing interessant.

        \item Beide sind sich selbst organisierende, infrastrukturlose Netzwerke. Unterschiede
        sind Ausrüstung, Anwendung, Umgebung, Größe, Instandhaltung,
        QoS, mobility und data-centric(SN) vs. id-centric(AN).

        \item Ein Anwendungsgebiet von Ad-hoc Netzwerken ist z.B. das Aufstellen eines Kommunikationsnetzwerk nach einer Naturkatastrophe, die die primäre Kommunikations-Infrastruktur beschädigt hat. Ein Anwendungsgebiet für Sensornetzwerke ist z.B. die überwachung kritischer Infrastruktur wie Brücken oder Dämmen auf Anzeichen von Materialermüdung oder übermäßigem Stress.
    \end{enumerate}

    \task{2}{13}
    \begin{enumerate}[label=\arabic*)]
        \item Das Paper erklärt den Aufbau und die Funktionsweise von Ad-hoc Netzwerken und deren Anwendungsgebiete. Weiter gibt es einen Einblick in die Entstehungsweise und Anfänge von Ad-hoc Netzwerken. Da es noch ein recht neues Gebiet ist, gibt es auch noch Probleme denen man nachgehen muss, von denen einige aufgezählt werden. Zuletzt wird ein Einblick die Zukunft von Ad-hoc Netzwerken gegeben und auch mögliche Wege wie sich dieses Gebiet weiterentwickeln könnte.
        
        \item \begin{itemize}
            \item 1972: DoD-sponsored Packet Radio Network (PRNET)
            \item 1980s: Survivable Adaptive Radio Network (SURAN)
            \item Das Ziel dieser Netzwerke war es, Paketvermittelnde Netzwerke auf dem Schlachtfeld ein zu setzen, in einer Feindlichen umgebung ohne bestehende Infrastruktur.
            \item 1990: Ad-hoc Netzwerke werden vom IEEE 802.11 Subkomitee diskutiert und erreichen somit die kommerzielle Welt. 
        \end{itemize}
        \pagebreak
        \item \begin{itemize}
            \item \textbf{Scalability:} Die Frage ob das Netzwerk auch bei größerer Anzahl an Nodes guten Service leisten kann.
            
            \item \textbf{Energy-efficiency:} Da viele Nodes mobil sind und deswegen die Energiezufuhr limitiert ist, sind energieeffiziente Lösungen vurzuziehen.
            
            \item \textbf{QoS:} Ein Netzwerk soll nicht nur funktionieren, sondern auch gut funktionieren. Die Qualität muss angemessen sein.
        \end{itemize}
        
        \item Die im Paper erwähnten "Layers" beziehen sich aufg das OSI-Modell.
        
        \item Physical-Layer, Network layer, Transport layer, Application layer.
    \end{enumerate}

    \pointsttl{21}

\end{document}