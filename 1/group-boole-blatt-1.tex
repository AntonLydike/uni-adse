\documentclass{article}
\usepackage{microtype}
\usepackage[utf8]{inputenc} 
\usepackage[a4paper, total={6in, 9.6in}]{geometry}
\usepackage[ngerman]{babel}
\usepackage{enumitem}
\usepackage{amsmath}
\usepackage{amssymb}
\usepackage{fancyhdr}
\usepackage{xcolor}
\usepackage{tikz}
\usepackage{pgfplots}
\usepackage{svg}
\usepackage{graphicx}

\widowpenalties=4 10000 10000 150 0

% header / footer style
\pagestyle{fancy}
\fancyhf{}
\rhead{Ad-hoc und Sensornetze  SS20}
\lhead{Daniel Maquet, Leopold Luley, Anton Lydike}
\rfoot{Seite \thepage}

% define some basic colors
\definecolor{greeen}{RGB}{34,139,34}
\newcommand\red[1]{\textcolor{red}{#1}}
\newcommand\green[1]{\textcolor{greeen}{#1}}
\newcommand\blue[1]{\textcolor{blue}{#1}}

\setlength{\parindent}{0pt}


% define a task 
\newcommand\task[2]{\par\bigskip\textbf{Aufgabe #1)\hfill \underline{\,\,\,\,\,\,}\,\,/#2p.}\par}
% and the total points for this sheet
\newcommand\pointsttl[1]{\par\bigskip\textbf{Gesamtpunkte: \hfill \underline{\,\,\,\,\,\,}\,\,/#1p.}}


\newcommand\cfgtitle[1]{\title{\vspace{-1.5cm}Übungsblatt #1\\%
\begin{large} Übungsgruppe Boole \end{large}} \lfoot{Übungsblatt #1}\cfoot{Übungsgruppe Boole}}
\author{Daniel Maquet, Leopold Luley, Anton Lydike}

% configure title to be sheet 1
\cfgtitle{1}
\date{Donnerstag 10.05.2020}

\begin{document}
    \maketitle
    \thispagestyle{fancy}

    \task{1}{8}
    \begin{enumerate}[label=\arabic*)]
        \item Organic computing beschäftigt sich mit selbständigen systemen also Systemen, die sich selbst konfigurieren und optimieren. Aber auch selbst probleme erkennen und lösen. Dafür müssen sie ihre Umgebung wahrnehmen können und bestimmte Ereignisse zu einem bestimmten Grad vorraussehen können. 

        \item Ad-hoc und Sensornetze sind Netzwerke aus vergleichsweise low-tech Geräten, die ein oft temporäres Netzwerk aufspannen. Oft können diese durch folgende Eigenschaften identifiziert werden:
            \begin{enumerate}[label=\alph*)]
                \item Ohne zentrale Infrastruktur
                \item Kabellose Kommunikation
                \item Teilnehmer sind in bewegung
                \item Teilnehmer sind autonom
                \item Hererogene Endgeräte
                \item Batteriebetriebene Geräte
                \item Sammeln und Auswerten von Sensordaten vor Ort
            \end{enumerate}
        Solche Netzwerke werden oft verwendet um Informationen in einem geographisch sehr begrenzten Gebiet zu sammeln. 
        \item Ad-hoc und Sensornetze müssen oft Selbstorganisiert sein, da sie eine Dezentrale struktur haben. Auch ist es wichtig, optimale Routen durch solche netzwerke zu finden, um Batterielaufzeit zu erhöhen. Da nicht immer Qualifiziertes/Trainiertes Personal vorhanden ist, sollten diese Netzwerke sich so weit wie möglich selbst konfigurieren und reparieren. All diese Aspekte machen sie aus der Perspektive von Organic Computing interessant.
        \item \emph{Ad-hoc} Netzwerke sind netzwerke, welche meist ohne vorexistierende Infrastruktur aufgebaut werden. Oft werden hierfür die Geräte der Teilnehmer selbst verwendet. \emph{Sensornetze} hingegen sind Netzwerke von Räumlich getrennten, autonomen Geräten, welche ihre umgebung mit hilfe von Sensoren kontrollieren. 
        \item Ein Anwendungsgebiet von Ad-hoc Netzwerken ist z.B. das Aufstellen eines Kommunikationsnetzwerk nach einer Naturkatastrophe, die die primäre Kommunikations-Infrastruktur beschädigt hat. Ein Anwendungsgebiet für Sensornetzwerke ist z.B. die überwachung kritischer Infrastruktur wie Brücken oder Dämmen auf Anzeichen von Materialermüdung oder übermäßigem Stress.
    \end{enumerate}

    \task{2}{13}
    \begin{enumerate}[label=\arabic*)]
        \item 123
        \item abc
        \item def
        \item efg
        \item hij
    \end{enumerate}


    \pointsttl{21}

\end{document}